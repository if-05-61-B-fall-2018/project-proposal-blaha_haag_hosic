\documentclass[12pt]{scrartcl}
\usepackage{ucs}
\usepackage[utf8x]{inputenc}
\usepackage[T1]{fontenc}
\usepackage[english]{babel}
\usepackage{setspace}
\usepackage{floatrow}
\usepackage[table]{xcolor}
\usepackage{graphicx}
\usepackage{lmodern}
\usepackage[automark]{scrpage2}
\usepackage{geometry}  
\usepackage{amssymb}
\usepackage{amsthm}
\usepackage{epstopdf}
\usepackage{caption}
\usepackage{floatrow}
\usepackage[table]{xcolor}
\renewcommand{\baselinestretch}{1.15} 
\newcolumntype{L}[1]{>{\raggedright\let\newline\\\arraybackslash\hspace{0pt}}m{#1}}
\pagestyle{scrheadings}
\clearscrheadfoot
\ihead[]{\headmark}
\ifoot[]{\author}
\ofoot[]{\pagemark}
\setheadsepline[\textwidth]{0.1pt}
\setkomafont{pageheadfoot}{\sffamily}
\setkomafont{pagenumber}{\bfseries}

\DeclareGraphicsRule{.tif}{png}{.png}{`convert #1 `dirname #1`/`basename #1 .tif`.png}

\title{Features}
\author{Johann Haag, Tarik Hosic und Simon Blaha}
\date{18.6.2019, Leonding}


\begin{document}
    \maketitle
    \begin{flushleft}
    \begin{tabular}{|l|l|}
    \hline
    Project Name & Smart Organizer \\ \hline
    Project Leader & Simon Blaha \\ \hline
    Version & 1.0\\ \hline
    Document state & In process \\ \hline
    \end{tabular}
    \end{flushleft}

    \pagebreak
    \tableofcontents
    \pagebreak

    \section{Feature List}
    \subsection{Rest-Server}
    \begin{itemize}
        \item Create appointment on Rest-Server
        \item Edit appointment on Rest-Server
        \item Delete appointment on Rest-Server
        \item Create group on Rest-Server
        \item Delete group on Rest-Server
        \item Get all groups from user
        \item Get all appointments from user
        \item Login user 
        \item Logout user 
        \item Check user
    \end{itemize}
    \subsection{CalendarGui}
    \begin{itemize}
        \item Create appointments
        \item Delete appointments
        \item Edit appointments
        \item Select appointments
        \item Show appointments
        \item Calendar Views
    \end{itemize}
    \subsection{Login}
    \begin{itemize}
        \item Benutzeroberfläche zum Einloggen und Registrieren
    \end{itemize}
    

    \section{Feature Descriptions}
    \subsection{Rest-Server}
    \subsubsection{Create appointment on Rest-Server}
        Per POST-Request kann man einen Termin an den Server schicken.
        Der Termin muss als JSON übermittelt werden. Des Weiteren ist ein gültiger Benutzername sowie Token zu übergeben. Wobei der Username 
        bereits indirekt durch den Termin übergeben wird.
        Wenn diese stimmen, so wird der Termin für den entsprechenden User auf dem Server gespeichert.
    
    \subsubsection{Edit appointment on Rest-Server}
        Es kann per POST-Request ein existierender Termin bearbeitet werden. Hierfür muss der Termin als JSON an den Server übermittelt werden.
        Des Weiteren sind noch Benutzername sowie Token zu übergeben. Wenn der übergebene Termin existiert und der Username sowie Token gültig ist, 
        wird die Änderung des Termins am Server durchgeführt. Bei dem übergebenen Termin muss es sich nicht um einen vollständiges Terminobjekt handelt, es 
        müssen nur die Felder ausgefüllt werden, die geändert werden sollen. Hierbei bildet nur das ID-Feld des Termins eine Ausnahme, weil dieses zum Suchen 
        des zu ändernden Termins verwendet wird.
        Falls ein Fehler auftreten sollte, so wird dieser als JSON an den POST-Request-Sender zurückgegeben.

    \subsubsection{Delete appointment on Rest-Server}
        Zur Verwendug des Features muss hierbei ein Username, Token sowie die ID des zu löschenden Termins übergeben werden.
        Auf dem Server wird überprüft, ob es einen Termin für den übergebenen User existiert und ob der übergebene Token gültig ist.
        Ist dies der Fall, so wird der Termin gelöscht.
        Falls ein Fehler auftreten sollte, so wird dieser als JSON an den POST-Request-Sender zurückgegeben.
    
    \subsubsection{Create group on Rest-Server}
        Auf dem Server kann per POST-Request eine Gruppe erstellt werden.
        Hierbei muss die zu erstellende Gruppe als JSON übertragen werden.
        Des Weiteren muss Username sowie Token übertragen werden. Der Username wird bereits indirekt durch 
        die angegebene Gruppe übermittelt.
        Mittels dieser Informationen wird überprüft, ob der User existiert und die Identität 
        per Token festgestellt. Es wird anschließend, falls die Überprüfungen erfolgreich waren,
        eine Gruppe für den User erstellt. Der Besitzter der Gruppe ist der Ersteller.
        Bei Fehlerfall bzw. Erfolg wird die entsprechende Rückmeldung an den POST-Request-Sender 
        als JSON zurückgeschickt.

    \subsubsection{Delete group on Rest-Server}
        Um dieses Feature zu nutzen muss per POST-Request die Gruppenname sowie Username und Token übergeben werden.
        Beim übergebenen User muss es sich um den Ersteller der Gruppe handeln.
        Falls diese Bedingungen erfüllt werden, wird die Gruppe gelöscht und per JSON der Erfolg zurückgegeben.
        Wenn das Löschen fehlschlägt, so wird eine Fehlermeldung als JSON zurückgegeben. 

    \subsubsection{Get all groups of user}
        Der Server antwortet auf einen POST-REQUEST. Bei dem Request muss per JSON ein Username als auch ein gültiger Token übergeben werden.
        Falls dieser User existiert, werden alle Gruppen zurückgegeben, bei denen der User Besitzer ist.
        
    \subsubsection{Get all appointments}
        Um diese Funktionalität zu verwenden, muss an den Rest-Server ein POST-Request mit Username und Token geschickt werden.
        Falls, der Username sowie Token gültig ist, so werden alle Termine als JSON an den Sender zurückgeschickt.
        Bei einem Fehlerfall wird als JSON der Fehler an den Sender zurückgegeben.
    
    \subsubsection{Login user}
        Um diese Funktionalität des Rest-Servers zu verwenden, muss mittels POST-Request eine 
        JSON-Datei mit gültigen Username und Passwort geschickt werden.
        Falls das Passwort als auch Username gültig waren, so wird ein Token zurückgeschickt.
        Bei Misserfolg wird eine Fehlermeldung als JSON an den Sender zurückgeschickt.

    \subsubsection{Logout user}
        Um diese Funktionalität des Rest-Servers in Anspruch zu nehmen, muss mittels POST-Request ein JSON mit Username 
        sowie Token geschickt werden. Wenn sich Token sowie Username als gültig herausstellen, wird 
        der Erfolg als JSON an den Sender zurückgeschickt. Bei Misserfolg wird eine Fehlermeldung an den Sender 
        geschickt.
    \subsubsection{Check user}
        Mit dieser Funktionalität des Servers kann festgestellt werden, ob ein Username sich bereits in Verwendung 
        befindet. Um die Funktionalität des Servers zu verwenden muss ein POST-Request mit dem zu überprüfenden 
        Usernamen übermittelt werden.
        Der Server gibt anschließend zurück, ob dieser Username frei ist oder sich bereits in Verwendung befindet.

    \subsection{CalendarGui}
    \subsubsection{Create appointments}
        Bei der CalendarGui gibt es einen Button namens AddEvent. Sobald man den Titel, die Beschreibung, die Start und
        die Endzeit des Termines eingegeben hat, drückt man einfach auf den Button AddEvent und unter der CalendarView erscheint
        dann von wann bis wann der Termin geht und der Titel des Termins.
        
    \subsubsection{Delete appointments}
        Um dieses Feature nutzen können, muss vorher ein Termin erstellt werden. Wie oben beschrieben muss man einen Termin
        anhand des Button AddEvent hinzufügen und sobald der Termin hinzugefügt wurde, drückt man auf den Termin hinauf und
        es erscheint dann kleines Fenster, wo Delete steht.    
        
    \subsubsection{Edit appointments}
            
        Um dieses Feature nutzen zu können, muss vorher ein Termin erstellt werden. Wie oben beschrieben muss man einen Termin
        anhand des Button AddEvent hinzufügen und sobald der Termin hinzugefügt wurde, drückt man auf den Termin hinauf und
        es erscheint dann kleines Fenster, wo Edit steht.   

    \subsubsection{Select appointments}
        Sobald man einen Termin hinzugefügt hat, kann man auf der CalendarView auf ein Datum klicken, wo ein Termin
        hinzugefügt worden ist und dieser erscheint unter der CalendarView. Der darunter erscheinende Termin kann dann 
        ausgewählt werden.
    
    \subsubsection{Show appointments}
        Sobald man einen Termin hinzugefügt hat, kann man auf der CalendarView auf ein Datum klicken, wo ein Termin
        hinzugefügt worden ist und dieser erscheint unter der CalendarView. Wenn bei einem Datum kein Termin hinzugefügt    
        worden ist oder kein Termin ausgewählt worden ist, dann erscheint unter der CalendarView NoEvents.
            
    \subsubsection{CalendarViews}
        Es gibt drei verschiedene Ansichten des Kalendars. Die erste Ansicht davon ist die Monatsansicht, wo alle Tage von
        einem Monatsansicht angezeigt werden. Die zweite Ansicht des Kalendars ist die Wochenansicht. Bei der Wochenansicht
        werden sieben Tage eines Monats angezeigt. Man kann dann mit einem Pfeil, der rechts über der der Kalendaransicht
        erscheint, zur nächsten Woche gehen. Die dritte Ansicht des Kalendars ist die Tagesansicht. Bei der Tagesansicht
        werden die Uhrzeiten vom sechs Uhr früh bis sechs Uhr abends eines Tages angezeigt. Man kann dann mit einem Pfeil, 
        der rechts über der der Kalendaransicht erscheint, zum nächsten Tag gehen.
    \subsection{Login}
    \subsubsection{Benutzeroberfläche zum Einloggen und Registrieren}
        Beim Login kann ein User einen Username und ein Passwort eingeben.
        Unter dem Username und dem Passwort ist ein button Senden, sobald man auf diesem hinaufdrückt sollten die eingegebenen
        Daten mit den Daten in der Datenbank verglichen werden und danach sollte er den Kalendar anzeigen.
%------------------------------------------------------------------------%
\end{document}