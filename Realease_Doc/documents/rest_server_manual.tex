\documentclass[12pt]{scrartcl}
\usepackage{ucs}
\usepackage[utf8x]{inputenc}
\usepackage[T1]{fontenc}
\usepackage[english]{babel}
\usepackage{setspace}
\usepackage{floatrow}
\usepackage[table]{xcolor}
\usepackage{graphicx}
\usepackage{lmodern}
\usepackage[automark]{scrpage2}
\usepackage{listings}
\usepackage{geometry}  
\usepackage{amssymb}
\usepackage{amsthm}
\usepackage{epstopdf}
\usepackage{caption}
\usepackage{floatrow}\usepackage{hyperref}
\usepackage[table]{xcolor}
\renewcommand{\baselinestretch}{1.15} 
\newcolumntype{L}[1]{>{\raggedright\let\newline\\\arraybackslash\hspace{0pt}}m{#1}}
\pagestyle{scrheadings}
\clearscrheadfoot
\ihead[]{\headmark}
\ifoot[]{\author}
\ofoot[]{\pagemark}
\setheadsepline[\textwidth]{0.1pt}
\setkomafont{pageheadfoot}{\sffamily}
\setkomafont{pagenumber}{\bfseries}

\DeclareGraphicsRule{.tif}{png}{.png}{`convert #1 `dirname #1`/`basename #1 .tif`.png}

\title{Manual for Rest-Server}
\author{Johann Haag, Tarik Hosic und Simon Blaha}
\date{20.12.2018, Leonding}


\begin{document}
    \maketitle
    \begin{flushleft}
    \begin{tabular}{|l|l|}
    \hline
    Project Name & Smart Organizer \\ \hline
    Project Leader & Simon Blaha \\ \hline
    Version & 1.0\\ \hline
    Document state & In process \\ \hline
    \end{tabular}
    \end{flushleft}

    \pagebreak
    \tableofcontents
    \pagebreak

    \section{Installation}
    \subsection{Aufsetzen der MongoDB}
        Eine MongoDB kann standardmäßig auf den Computer installiert werden. Es bietet sich auch die Möglichkeit an, die MongoDB per Docker zu betreiben.
        Wenn sie die MongoDB auf den Rechner installieren möchten, sehen sie auf folgender Seite nach \url{https://www.mongodb.com}.
        Falls sie Docker verwenden möchten, dann sehen sie auf folgender Seite nach \url{https://hub.docker.com/_/mongo}.

    \subsection{Installation der benötigten Pakete für den Rest-Server}
        Wechseln Sie in das Root-Verzeichnis des Rest-Servers und geben Sie im Terminal npm i ein.
        Somit werden alle benötigten Pakete installiert.
        Informationen zu npm finden Sie auf \url{https://www.npmjs.com/get-npm}.
        Für den Betrieb des Servers wird node verwendet, Informationen darüber finden Sie auf \url{https://nodejs.org/}

    \section{Konfiguration}
        In der /database/conifg.json Datei des Servers können Datenbankeinstellungen des Servers geändert werden.
        Hier kann unter dbURI eine andere URI angegeben werden, falls man keinen Localhost verwendet.
        Bei der Standardinstallation ist immer Localhost und der standard Port für MongoDB eingetragen.
        Des Weiteren ist es auch möglich, die Fehlerausgaben zu ändern.

    \section{Inbetriebnahme des Rest-Servers}
        Um den Server zu starten gibt es mehrere Wege. 
        Zum einen können sie per CMD in das Root-Verzeichnis des Servers wechseln und node index.js ausführen oder 
        Sie führen das Script run aus. Hierbei wird jedoch die Skriptsprache Bash benötigt.
        Bei der Inbetriebnahme werden automatisch die benötigten Collections auf der MongoDB erstellt. 
        Falls alles korrekt gestartet wird, wird ausgegeben, dass der Server auf dem Port 8080 lauscht
        und dass eine Verbindung zur MongoDB hergestellt wurde.
        Der Server kann einfach mit STRG+C im Terminal beendet werden.    
    \section{Verwendung des Servers}
        Es wird davon ausgegangen, dass der Server auf Localhost läuft.
        Falls etwas schiefläuft, wird immer ein JSON-Objekt mit den Attributen error und code zurückgegeben.
        Code besitzt den Wert des Fehlercodes und im Attribut error steht die Fehlermeldung.
        Wenn eine Abfrage erfolgreich ist, so wird ein JSON-Objekt mit einem Attribut result zurückgegeben.
    \subsection{User registrieren}
        Registriert einen neuen User auf dem Server.
        Hierfür muss ein JSON-Objekt per Post-Request an den Server geschickt werden. Der Pfad für den Post-Request lautet
        \url{http://localhost:8080/user/register}.
        Struktur des JSON-Objekts: \{"username":"<username>", "password":"<password>"\}
        Das Passwort wird auf dem Server verschlüsselt gespeichert.

    \subsection{User login}
        Einloggen eines Users auf dem Server. Beim Login wird ein Token für den Benutzer angelegt.
        Hierfür muss ein JSON-Objekt per Post-Request an den Server geschickt werden. Der Pfad für den Post-Request lautet
        \url{http://localhost:8080/user/login}.
        Struktur des JSON-Objekts: \{"username":"<username>", "password":"<password>"\}
        Bei erfolgreichen Login wird ein Token zurückgegeben. Der Token wird als Berechtigung für einige Funktionen auf dem Server benötigt.
    
    \subsection{Logout}
        Ausloggen eines Users auf dem Server.
        Hierfür muss ein JSON-Objekt per Post-Request an den Server geschickt werden. Der Pfad für den Post-Request lautet
        \url{http://localhost:8080/user/logout}.
        Struktur des JSON-Objekts: \{"username":"<username>", "token":"<token>"\}
        Wenn dies erfolgreich ist, so gibt es für den Benutzer keinen gültigen Token mehr.

    \subsection{Check if user exists}
        Überprüfen, ob ein Username bereits vergeben ist.
        Hierfür muss ein JSON-Objekt per Post-Request an den Server geschickt werden. Der Pfad für den Post-Request lautet
        \url{http://localhost:8080/user/check}.
        Struktur des JSON-Objekts: \{"username":"<username>"\}

    \subsection{Create Appointment}
        Erstellen eines Termins für einen User.
        Hierfür muss ein JSON-Objekt per Post-Request an den Server geschickt werden. Der Pfad für den Post-Request lautet
        \url{http://localhost:8080/appoitnment/create}.
        Struktur des JSON-Objects: \{"appointment": \{"username": "<username>", "date": "<date>", "duration": "<number>", "name": "<name>"\}, "token": "<token>"\}
        Der Token muss ein gültiger Token des Benutzers sein, dessen Username im Appointment verwendet wurde.

    \subsection{Delete Appointment}
        Löschen eines Termines eines Users.
        Hierfür muss ein JSON-Objekt per Post-Request an den Server geschickt werden. Der Pfad für den Post-Request lautet
        \url{http://localhost:8080/appoitnment/delete}.
        Struktur des JSON-Objects: \{"username": "<username>", "token": "<token>", "appointmentid": "<appointmentid>" \}
        Der übergebene Username muss der Besitzer des Termines sein sowie der Token muss gültig sein.

    \subsection{Edit Appointment}
        Editieren eines Termins eines Users.
        Hierfür muss ein JSON-Objekt per Post-Request an den Server geschickt werden. Der Pfad für den Post-Request lautet \url{http://localhost:8080/appoitnment/edit}.
        Struktur des JSON-Objects: \{"appointment": {"username": "<username>"[, "date": "<date>"][, "duration": "<number>"][, "name": "<name>"], "token": "<token>"\}
        Die in {[]} geschriebenen Attribute sind optional!
    

    \subsection{Get all appointments}
        Alle Termine eines Users abfragen.
        Hierfür muss ein JSON-Objekt per Post-Request an den Server geschickt werden. Der Pfad für den Post-Request lautet
        http://localhost:8080/appoitnment/get.
        Struktur des JSON-Objects: { "username": "<username>", "token": "<token>"}

    \subsection{Create group}
        Eine Gruppe erstellen.
        Hierfür muss ein JSON-Objekt per Post-Request an den Server geschickt werden. Der Pfad für den Post-Request lautet
        \url{http://localhost:8080/group/create}.
        Struktur des JSON-Objects: {"group": {"owner": "<username>", "name": "<groupName>" }, "token": "<token>"}
        Der übergebene Token muss ein gültiger Token für den übergebenen username sein.

    \subsection{Delete group}
        Eine Gruppe erstellen.
        Hierfür muss ein JSON-Objekt per Post-Request an den Server geschickt werden. Der Pfad für den Post-Request lautet
        \url{http://localhost:8080/group/delete}.
        Struktur des JSON-Objects: {"group": {"owner": "<username>", "name": "<groupName>" }, "token": "<token>"}

    \subsection{Get groups}
        Alle Gruppen bekommen für den der User der Besitzer ist.
        Hierfür muss ein JSON-Objekt per Post-Request an den Server geschickt werden. Der Pfad für den Post-Request lautet
        \url{http://localhost:8080/group/get}.
        Struktur des JSON-Objects: \{"username": "<username>", "token": "<token>"\}
\end{document}