\documentclass[12pt]{scrartcl}
\usepackage{ucs}
\usepackage[utf8x]{inputenc}
\usepackage[T1]{fontenc}
\usepackage[english]{babel}
\usepackage{setspace}
\usepackage{floatrow}
\usepackage[table]{xcolor}
\usepackage{graphicx}
\usepackage{lmodern}
\usepackage[automark]{scrpage2}
\usepackage{listings}
\usepackage{geometry}  
\usepackage{amssymb}
\usepackage{amsthm}
\usepackage{epstopdf}
\usepackage{caption}
\usepackage{floatrow}\usepackage{hyperref}
\usepackage[table]{xcolor}
\renewcommand{\baselinestretch}{1.15} 
\newcolumntype{L}[1]{>{\raggedright\let\newline\\\arraybackslash\hspace{0pt}}m{#1}}
\pagestyle{scrheadings}
\clearscrheadfoot
\ihead[]{\headmark}
\ifoot[]{\author}
\ofoot[]{\pagemark}
\setheadsepline[\textwidth]{0.1pt}
\setkomafont{pageheadfoot}{\sffamily}
\setkomafont{pagenumber}{\bfseries}

\DeclareGraphicsRule{.tif}{png}{.png}{`convert #1 `dirname #1`/`basename #1 .tif`.png}

\title{Manual for REST Server}
\author{Johann Haag, Tarik Hosic und Simon Blaha}
\date{18.6.2019, Leonding}


\begin{document}
    \maketitle
    \begin{flushleft}
    \begin{tabular}{|l|l|}
    \hline
    Project Name & Smart Organizer \\ \hline
    Project Leader & Simon Blaha \\ \hline
    Version & 1.0\\ \hline
    Document state & In process \\ \hline
    \end{tabular}
    \end{flushleft}

    \pagebreak
    \tableofcontents
    \pagebreak

    \section{Installation}
    \subsection{Setting up the MongoDB}
        A MongoDB can be installed on the computer by default. It is also possible to operate the MongoDB via Docker.
        If you want to install MongoDB on your computer, you will see \url{https://www.mongodb.com} 
        If you want to use Docker, then look at the followng page \url{https://hub.docker.com/_/mongo}.
    \subsection{Installation of the required packages for the REST server}
        Change to the root directory of the REST server and enter npm i in the terminal.
        Thus, all required packages are installed.
        For information about npm, see \url{https://www.npmjs.com/get-npm}.
        Node is used to operate the server, for information about node see \url{https://nodejs.org/}.

    \section{Configuration}      
        In the server's /database/conifg.json file, database settings of the server can be changed.
        Here you can specify a different URI under dbURI if you do not use a localhost.
        The default installation always includes localhost and the default port for MongoDB.
        Furthermore, it is also possible to change the error messages.

    \section{Commissioning of the REST Server}
        There are several ways to start the server.
        On the one hand, they can switch to the root directory of the server via CMD and execute node index.js or
        you run the script run. However, this requires the scripting language Bash.
        During commissioning, the required collections are automatically created on the MongoDB.
        If everything starts up correctly, the server will listen on port 8080
        and that a connection to MongoDB was made.
        The server can be easily terminated with STRG + C in the terminal.
    \section{Using the REST Server}
        It is assumed that the server is running on localhost.
        If something goes wrong, it will always return a JSON object with the error and code attributes.
        Code has the value of the error code and the attribute error contains the error message.
        If a query succeeds, a JSON object with a result attribute is returned.
    \subsection{User Registration}
        Create a new user on the server.
        To do this, a JSON object must be sent to the server by POST request. The path for the POST request is
        {http://localhost:8080/user/register}.
        Structure of the JSON object: \{"username": "<username>", "password": "<password>"\}
        The password is stored encrypted on the server.

    \subsection{User Login}
        Login of a user on the server. When logging in, a token is created for the user.
        To do this, a JSON object must be sent to the server by POST request. The path for the POST request is
        {http://localhost:8080/user/login}.
        Structure of the JSON object: \{"username": "<username>", "password": "<password>"\}
        Upon successful login, a token is returned. The token is required as authorization for some functions on the server.
    
    \subsection{User Logout}
        Logging out a user on the server.
        To do this, a JSON object must be sent to the server by POST request. The path for the POST request is
        {http://localhost:8080/user/logout}.
        Structure of the JSON object: \{"username": "<username>", "token": "<token>"\}
        If successful, the user will no longer have a valid token.

    \subsection{Check if User exists}
        Check if a username is already taken.
        To do this, a JSON object must be sent to the server by POST request. The path for the POST request is
        {http://localhost:8080/user/check}.
        Structure of the JSON object: \{"username": "<username>"\}

    \subsection{Create Appointment}
        Create an appointment for a user.
        To do this, a JSON object must be sent to the server by POST request. The path for the POST request is
        {http://localhost:8080/appoimentnt/create}.
        Structure of the JSON object: \{"appointment": \{"username": "<username>", "date": "<date>", "duration": <number>, "name": "<name>"\},"token" : "<token>"\}
        The token must be a valid token of the user whose username was used in the appointment.

    \subsection{Delete Appointment}
        Delete an appointment of a user.
        To do this, a JSON object must be sent to the server by POST request. The path for the POST request is
        {http://localhost:8080/appoitnment/delete}.
        Structure of the JSON object: \{"username": "<username>", "token": "<token>", "appointmentid": "<appointmentid>"\}
        The transferred username must be the owner of the appointment and the token must be valid.

    \subsection{Edit Appointment}
        Editing an appointment of a user.
        To do this, a JSON object must be sent to the server by POST request. The path for the POST request is {http://localhost:8080/appoitnment/edit}.
        Structure of the JSON object: \{"appointment": \{"username": "<username>" [, "date": "<date>"] [, "duration": "<number>"] [, "name ":" <name> "]\},"token" :"<token>"\}
        The attributes written in {[ ]} are optional!
    

    \subsection{Get all Appointments}
        Query all appointments of a user.
        To do this, a JSON object must be sent to the server by POST request. The path for the POST request is
        {http://localhost:8080/appoitnment/get}.
        Structure of the JSON object: \{"username": "<username>", "token": "<token>"\}

    \subsection{Create Group}
        Create a group for a user.
        To do this, a JSON object must be sent to the server by POST request. The path for the POST request is
        {http://localhost:8080/group/create}.
        Structure of the JSON object: \{"group": \{"owner": "<username>", "name": "<groupName>"\}, "token": "<token>"\}
        The passed token must be a valid token for the given username.

    \subsection{Delete Group}
        To do this, a JSON object must be sent to the server by POST request. The path for the POST request is
        {http://localhost:8080/group/delete}.
        Structure of the JSON object: \{"group": \{"owner": "<username>", "name": "<groupName>"\}, "token": "<token>"\}

    \subsection{Get Groups}
        Get all group where the user is the owner of it.
        To do this, a JSON object must be sent to the server by POST request. The path for the POST request is
        {http://localhost:8080/group/get}.
        Structure of the JSON object: \{"username": "<username>", "token": "<token>"\}
\end{document}