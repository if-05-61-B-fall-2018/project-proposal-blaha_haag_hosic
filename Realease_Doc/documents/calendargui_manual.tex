\documentclass[12pt]{scrartcl}
\usepackage{ucs}
\usepackage[utf8x]{inputenc}
\usepackage[T1]{fontenc}
\usepackage[english]{babel}
\usepackage{setspace}
\usepackage{floatrow}
\usepackage[table]{xcolor}
\usepackage{graphicx}
\usepackage{lmodern}
\usepackage[automark]{scrpage2}
\usepackage{geometry}  
\usepackage{amssymb}
\usepackage{amsthm}
\usepackage{epstopdf}
\usepackage{caption}
\usepackage{floatrow}
\usepackage[table]{xcolor}
\renewcommand{\baselinestretch}{1.15} 
\newcolumntype{L}[1]{>{\raggedright\let\newline\\\arraybackslash\hspace{0pt}}m{#1}}
\pagestyle{scrheadings}
\clearscrheadfoot
\ihead[]{\headmark}
\ifoot[]{\author}
\ofoot[]{\pagemark}
\setheadsepline[\textwidth]{0.1pt}
\setkomafont{pageheadfoot}{\sffamily}
\setkomafont{pagenumber}{\bfseries}

\DeclareGraphicsRule{.tif}{png}{.png}{`convert #1 `dirname #1`/`basename #1 .tif`.png}

\title{Manual for CalendarGui}
\author{Johann Haag, Tarik Hosic und Simon Blaha}
\date{18.6.2019, Leonding}


\begin{document}
    \maketitle
    \begin{flushleft}
    \begin{tabular}{|l|l|}
    \hline
    Project Name & Smart Organizer \\ \hline
    Project Leader & Simon Blaha \\ \hline
    Version & 1.0\\ \hline
    Document state & In process \\ \hline
    \end{tabular}
    \end{flushleft}

    \pagebreak
    \tableofcontents
    \pagebreak

    \section{Installation}
    \subsection{Creating the Ionic-App}
        First you have to install NodeJS.
        Then enter the npm install -g ionic command in the console to install Ionic.
        Finally, enter into the console ionic start myApp tabs to create a new Ionic app called myApp.

    \section{Testing the Installation}
        In the console, type cd myApp to get to the directory.
        Then enter ionic serve --open.
        Finally, the created Ionic App will be displayed in the web browser.
%------------------------------------------------------------------------%
\end{document}