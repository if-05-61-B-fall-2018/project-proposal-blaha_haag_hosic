\documentclass[12pt]{scrartcl}
\usepackage{ucs}
\usepackage[utf8x]{inputenc}
\usepackage[T1]{fontenc}
\usepackage[english]{babel}
\usepackage{setspace}
\usepackage{floatrow}
\usepackage[table]{xcolor}
\usepackage{graphicx}
\usepackage{lmodern}
\usepackage[automark]{scrpage2}
\usepackage{geometry}  
\usepackage{amssymb}
\usepackage{amsthm}
\usepackage{epstopdf}
\usepackage{caption}
\usepackage{floatrow}
\usepackage[table]{xcolor}
\renewcommand{\baselinestretch}{1.15} 
\newcolumntype{L}[1]{>{\raggedright\let\newline\\\arraybackslash\hspace{0pt}}m{#1}}
\pagestyle{scrheadings}
\clearscrheadfoot
\ihead[]{\headmark}
\ifoot[]{\author}
\ofoot[]{\pagemark}
\setheadsepline[\textwidth]{0.1pt}
\setkomafont{pageheadfoot}{\sffamily}
\setkomafont{pagenumber}{\bfseries}


\title{Smart Personal Organizer}
\author{Johann Haag, Tarik Hosic und Simon Blaha}
\date{18.6.2019, Leonding}


\begin{document}
    \maketitle
    \begin{flushleft}
    \begin{tabular}{|l|l|}
    \hline
    Project Name & Smart Organizer \\ \hline
    Project Leader & Simon Blaha \\ \hline
    Version & 1.0\\ \hline
    Document state & In process \\ \hline
    \end{tabular}
    \end{flushleft}

    \pagebreak
    \tableofcontents
    \pagebreak


    \section{Tests}                             
    \subsection{REST Server}
    \subsubsection{Test 1 - Register User}
        \begin{tabular}{|L{2cm}|L{7cm}|L{6cm}|} 
            \hline 
            \cellcolor[gray]{0.5}\textcolor{white}{Test step} & \cellcolor[gray]{0.5}\textcolor{white}{Description} & \cellcolor[gray]{0.5}\textcolor{white}{Expected Result} \\ \hline
            1 & Register non existing user on REST server per POST request and with valid JSON-Object & The user will be created on the server and the server send a JSON-Object with the username of the created user.\\  \hline
            2 & Register an existing user on REST server per POST request and with valid JSON-Object & The user will not be created on the server send an error with error code back.\\  \hline
            3 & Send incorrect JSON object to the REST server per Post request & The server answers with a JSON-Object with an error (Unvalid request).\\  \hline
        \end{tabular}

    \subsubsection{Test 2 - Login User}
        \begin{tabular}{|L{2cm}|L{7cm}|L{6cm}|} 
            \hline 
            \cellcolor[gray]{0.5}\textcolor{white}{Test step} & \cellcolor[gray]{0.5}\textcolor{white}{Description} & \cellcolor[gray]{0.5}\textcolor{white}{Expected Result} \\ \hline
            1 & Login existing user with correct password and username & The server will answer with an JSON-Object with a username attribut and a token attribut.\\  \hline
            2 & Try to login non existing user on REST server per POST request & The server answers with a JSON-Object with an error and error code.\\  \hline
            3 & Try to login an existing user on REST server per POST request with wrong password & The server answers with a JSON-Object with an error and error code.\\  \hline
            4 & Send JSON object with too many/less attributes to the REST server per Post request & The server answers with a JSON-Object with an error (Unvalid request).\\  \hline
        \end{tabular}

    \subsubsection{Test 3 - Logout User}
        \begin{tabular}{|L{2cm}|L{7cm}|L{6cm}|} 
            \hline 
            \cellcolor[gray]{0.5}\textcolor{white}{Test step} & \cellcolor[gray]{0.5}\textcolor{white}{Description} & \cellcolor[gray]{0.5}\textcolor{white}{Expected Result} \\ \hline
            1 & Logout existing user with correct token and username & The server will answer with a JSON-Object with result that the result logout.\\  \hline
            2 & Try to logout non existing user on REST server per POST request & The server answers with a JSON-Object with an error and error code.\\  \hline
            3 & Try to logout an existing user on REST server per POST request with wrong token & The server answers with a JSON-Object with an error and error code.\\  \hline
            4 & Send JSON object with too many/less attributes to the REST server per Post request & The server answers with a JSON-Object with an error (Unvalid request).\\  \hline
        \end{tabular}

    \subsubsection{Test 4 - Create Appointment}
        \begin{tabular}{|L{2cm}|L{7cm}|L{6cm}|} 
            \hline 
            \cellcolor[gray]{0.5}\textcolor{white}{Test step} & \cellcolor[gray]{0.5}\textcolor{white}{Description} & \cellcolor[gray]{0.5}\textcolor{white}{Expected Result} \\ \hline
            1 & Create appointment with valid token and username & The server creates the appointment and gives back a result.\\  \hline
            2 & Create appoitnment with invalid token or invalid username & The server sends an error.\\  \hline
            3 & Send JSON object with too less attributes to the REST server per Post request & The server answers with "Unvalid request" error.\\  \hline
        \end{tabular}

    \subsubsection{Test 5 - Delete appointment}
        \begin{tabular}{|L{2cm}|L{7cm}|L{6cm}|} 
            \hline 
            \cellcolor[gray]{0.5}\textcolor{white}{Test step} & \cellcolor[gray]{0.5}\textcolor{white}{Description} & \cellcolor[gray]{0.5}\textcolor{white}{Expected Result} \\ \hline
            1 & Send valid JSON-Object with valid username and token and appointmentid & The server will answer with a result.\\  \hline
            2 & Send valid JSON-Object with invalid username or token & The server will answer with an error that username or token is not correct.\\  \hline
            3 & Send JSON object with too less attributes to the REST server per Post request & The server answers with "Unvalid request" error.\\  \hline
        \end{tabular}

    \subsubsection{Test 6 - Edit appointment}
        \begin{tabular}{|L{2cm}|L{7cm}|L{6cm}|} 
            \hline 
            \cellcolor[gray]{0.5}\textcolor{white}{Test step} & \cellcolor[gray]{0.5}\textcolor{white}{Description} & \cellcolor[gray]{0.5}\textcolor{white}{Expected Result} \\ \hline
            1 & Edit existing appointment with valid JSON-Object & The server will answer with a result.\\  \hline
            2 & Send JSON object with too less attributes to the REST server per Post request & The server answers with "Unvalid request" error.\\  \hline
            3 & Edit existing appointment with valid JSON-Object but the user is not the owner of the appointment& The server will answer with an error.\\  \hline
            4 & Edit non existing appointment & The server will answer with an error.\\  \hline
        \end{tabular}

    \subsubsection{Test 7 - Get all appointments}
        \begin{tabular}{|L{2cm}|L{7cm}|L{6cm}|} 
            \hline 
            \cellcolor[gray]{0.5}\textcolor{white}{Test step} & \cellcolor[gray]{0.5}\textcolor{white}{Description} & \cellcolor[gray]{0.5}\textcolor{white}{Expected Result} \\ \hline
            1 & Send valid JSON object with valid token and username & The server will send all appointments of the given user.\\  \hline
            2 & Send valid JSON object with invalid token or invalid username & The server will send an error.\\  \hline
            3 & Send JSON object with too less attributes to the REST server per Post request & The server answers with "Unvalid request" error.\\  \hline
        \end{tabular}

    \subsubsection{Test 8 - Create Group}
        \begin{tabular}{|L{2cm}|L{7cm}|L{6cm}|} 
            \hline 
            \cellcolor[gray]{0.5}\textcolor{white}{Test step} & \cellcolor[gray]{0.5}\textcolor{white}{Description} & \cellcolor[gray]{0.5}\textcolor{white}{Expected Result} \\ \hline
            1 & Send valid JSON object with valid token and username and valid name & The server will send a result and will create the group.\\  \hline
            2 & Send valid JSON object with invalid token or invalid username or existing name & The server will send an error.\\  \hline
            3 & Send JSON object with too less attributes to the REST server per Post request & The server answers with "Unvalid request" error.\\  \hline
        \end{tabular}

    \subsubsection{Test 9 - Delete Group}
        \begin{tabular}{|L{2cm}|L{7cm}|L{6cm}|} 
            \hline 
            \cellcolor[gray]{0.5}\textcolor{white}{Test step} & \cellcolor[gray]{0.5}\textcolor{white}{Description} & \cellcolor[gray]{0.5}\textcolor{white}{Expected Result} \\ \hline
            1 & Send valid JSON object with valid token and username and valid name and valid groupname & The server will send a result and will delete the group.\\  \hline
            2 & Send valid JSON object with invalid token or invalid username or existing name & The server will send an error.\\  \hline
            3 & Send JSON object with too less attributes to the REST server per Post request & The server answers with "Unvalid request" error.\\  \hline
        \end{tabular}

    \subsubsection{Test 10 - Get Group}
        \begin{tabular}{|L{2cm}|L{7cm}|L{6cm}|} 
            \hline 
            \cellcolor[gray]{0.5}\textcolor{white}{Test step} & \cellcolor[gray]{0.5}\textcolor{white}{Description} & \cellcolor[gray]{0.5}\textcolor{white}{Expected Result} \\ \hline
            1 & Send valid username and valid token & The server will send all groups in which the user is the owner. \\  \hline
            2 & Send JSON object with too less attributes to the REST server per Post request & The server answers with "Unvalid request" error. \\  \hline
            3 & Send invalid username or invalid token & The server will answer with an error. \\ \hline
        \end{tabular}

    \subsubsection{Test 11 - Path}
        \begin{tabular}{|L{2cm}|L{7cm}|L{6cm}|} 
            \hline 
            \cellcolor[gray]{0.5}\textcolor{white}{Test step} & \cellcolor[gray]{0.5}\textcolor{white}{Description} & \cellcolor[gray]{0.5}\textcolor{white}{Expected Result} \\ \hline
            1 & Send a JSON object to a non existing path & The server will answer with an error and error code. \\ \hline
        \end{tabular}
                           
    \subsection{CalendarGUI}
    \subsubsection{Test 1/ Create appointments}
        \begin{tabular}{|L{2cm}|L{7cm}|L{6cm}|} 
            \hline 
            \cellcolor[gray]{0.5}\textcolor{white}{Test step} & \cellcolor[gray]{0.5}\textcolor{white}{Description} & \cellcolor[gray]{0.5}\textcolor{white}{Expected Result} \\ \hline
            1 & Once you have the title, the description, the start and has entered the end time of the appointment, simply press the button AddEvent. & As soon as you press the Add button, the CalendarView will appear then from when to when the appointment is and the title of the appointment. \\  \hline
        \end{tabular}
    \subsubsection{Test 2/ Delete appointments}
        \begin{tabular}{|L{2cm}|L{7cm}|L{6cm}|} 
            \hline 
            \cellcolor[gray]{0.5}\textcolor{white}{Test step} & \cellcolor[gray]{0.5}\textcolor{white}{Description} & \cellcolor[gray]{0.5}\textcolor{white}{Expected Result} \\ \hline
            1 & As described above you have an appointment using the AddEvent button, and once the appointment has been added, press the appointment and it then appears small window where Delete stands. &  As soon as you press the Delete button, the appointment is removed. \\  \hline
        \end{tabular}
    \subsubsection{Test 3/ Edit appointments}
        \begin{tabular}{|L{2cm}|L{7cm}|L{6cm}|} 
            \hline 
            \cellcolor[gray]{0.5}\textcolor{white}{Test step} & \cellcolor[gray]{0.5}\textcolor{white}{Description} & \cellcolor[gray]{0.5}\textcolor{white}{Expected Result} \\ \hline
            1 & As described above you have an appointment using the AddEvent button, and once the appointment has been added, press the appointment andit then appears small window where Edit is. &  As soon as you press the edit button, the date will not be edited. The test fails. \\  \hline
        \end{tabular}
    \subsubsection{Test 4/ Select appointments}
        \begin{tabular}{|L{2cm}|L{7cm}|L{6cm}|} 
            \hline 
            \cellcolor[gray]{0.5}\textcolor{white}{Test step} & \cellcolor[gray]{0.5}\textcolor{white}{Description} & \cellcolor[gray]{0.5}\textcolor{white}{Expected Result} \\ \hline
            1 & Once you have added an appointment, you can click on CalendarView on a date where an appointment has been added and this appears under the CalendarView. & The date appearing below can then to be selected. \\  \hline
        \end{tabular}
    \subsubsection{Test 5/ Show appointments}
        \begin{tabular}{|L{2cm}|L{7cm}|L{6cm}|} 
            \hline 
            \cellcolor[gray]{0.5}\textcolor{white}{Test step} & \cellcolor[gray]{0.5}\textcolor{white}{Description} & \cellcolor[gray]{0.5}\textcolor{white}{Expected Result} \\ \hline
            1 & Once you have added an appointment, you can click on CalendarView on a date where an appointment has been added and this appears under the CalendarView.& passed \\  \hline
        \end{tabular}
    \subsubsection{Test 6/ CalendarViews}
        \begin{tabular}{|L{2cm}|L{7cm}|L{6cm}|} 
            \hline 
            \cellcolor[gray]{0.5}\textcolor{white}{Test step} & \cellcolor[gray]{0.5}\textcolor{white}{Description} & \cellcolor[gray]{0.5}\textcolor{white}{Expected Result} \\ \hline
            1 & There are three different views of the calendar. The first view of this is the month view, where all days of a month view. The second view of the calendar is the week view. In the week view are displayed seven days of a month. You can then use an arrow that is right above the calendar view appears, go to next week. The third view of the calendar is the day view. In the day view The times are displayed from six o'clock in the morning until six in the evening of one day. You can then use an arrow,which appears on the right above the calendar view, go to the next day. & passed \\  \hline
        \end{tabular}
    
\end{document}